\section{Introducción}
La robótica es una disciplina fundamental dentro de la Ingeniería en Mecatrónica, ya que permite el diseño, modelado y control de sistemas automatizados capaces de interactuar con su entorno. Uno de los conceptos clave en esta área es la cinemática, que se divide en cinemática directa e inversa. La cinemática directa permite determinar la posición y orientación del efector final de un robot a partir de los ángulos articulares, mientras que la cinemática inversa resuelve el problema inverso: encontrar los valores articulares necesarios para que el robot alcance una posición y orientación deseadas.

En este reporte se presenta el desarrollo de una práctica realizada en MATLAB, en la cual se empleó el modelo de Denavit-Hartenberg para describir la geometría del manipulador robótico. Posteriormente, se implementó un código de cinemática inversa con el fin de simular el movimiento del robot desde una posición inicial hasta una posición objetivo. Además, se generaron gráficas y animaciones que permiten visualizar la trayectoria seguida por el robot, así como los cambios en los ángulos articulares a lo largo del tiempo.

Esta actividad tiene como propósito reforzar el entendimiento del modelado cinemático y la programación de trayectorias en robots, habilidades esenciales para el desarrollo de soluciones automatizadas en el ámbito industrial y de investigación.