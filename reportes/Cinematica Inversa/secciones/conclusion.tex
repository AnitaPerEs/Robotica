\section{Conclusión}
A través del desarrollo de este trabajo se logró implementar con éxito un modelo cinemático para los robots R6 y R7, aplicando los parámetros de Denavit-Hartenberg y resolviendo la cinemática inversa mediante programación en MATLAB. Las gráficas obtenidas permitieron visualizar tanto el comportamiento de las articulaciones en el espacio cartesiano, lo cual facilitó la validación del modelo y del algoritmo propuesto.

El análisis de la cinemática articular mostró cómo cada articulación contribuye al movimiento general del robot, mientras que la cinemática cartesiana permitió observar si el efector final seguía correctamente la trayectoria deseada. En ambos casos, los resultados fueron consistentes con las expectativas teóricas. La gráfica del error de seguimiento evidenció un buen desempeño del sistema, con errores mínimos en la mayoría de las muestras analizadas.

No obstante, se presentaron algunas situaciones puntuales en las que el algoritmo de cinemática inversa no alcanzó la tolerancia deseada, resultando en errores más elevados. Esto sugiere que, aunque el método es efectivo en general, puede requerir ajustes o estrategias adicionales para mejorar la convergencia en casos particulares.

Esta experiencia permitió reforzar de manera práctica los conceptos de cinemática directa e inversa, y subraya la importancia de una correcta parametrización y programación para el control eficiente de manipuladores robóticos. Sin duda, estos conocimientos son fundamentales para el desarrollo de soluciones mecatrónicas aplicadas a la automatización y la industria.







