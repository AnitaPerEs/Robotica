\section{Conclusión}
En la elaboración de este reporte, se utilizó por primera vez el software LaTeX para la edición del documento y Sourcetree para la gestión del control de movimientos entre modificaciones de datos del documento por parte de los distintos miembros del equipo. A lo largo del proceso, se presentaron dificultades relacionadas con el uso de Sourcetree, especialmente en la ejecución de comandos como branches, commit, push y pull, lo que requirió una curva de aprendizaje adicional para comprender su funcionamiento y evitar conflictos en la sincronización de archivos.

A pesar de estos desafíos, la experiencia permitió familiarizarse con herramientas clave para la edición y gestión colaborativa de documentos, lo que resultará beneficioso en futuros proyectos. La combinación de LaTeX y Sourcetree demostró ser una opción robusta para la elaboración de reportes técnicos con un control de organización eficiente, aunque es recomendable seguir profundizando en su uso para optimizar los flujos de trabajo y minimizar errores.

Por otra parte, el conocimiento más importante obtenido de la investigación sobre los distintos tipos de sensores fue la comprensión de cómo cada tecnología responde a necesidades específicas en diversas industrias. Esta investigación permitió no solo identificar las aplicaciones clave de cada sensor, sino también comprender la importancia de la selección adecuada según el entorno y el tipo de medición requerida. La evolución de los sensores, junto con su integración con inteligencia artificial y procesamiento de datos, sigue impulsando innovaciones en automatización, control y análisis del entorno.







