\chapter{Introducción} \label{chap:introduccion}

En este reporte se documenta el desarrollo de un proyecto centrado en la simulación y análisis de un brazo robótico. El objetivo principal fue comprender, implementar y validar distintos conceptos fundamentales en el campo de la robótica, tales como la cinemática directa e inversa, dinámica, control y simulación en entornos virtuales.

A lo largo del proyecto se utilizaron distintas herramientas, entre ellas Gazebo para la simulación 3D del entorno y del robot, ROS (Robot Operating System) para la integración y control modular del sistema, y MATLAB, el cual se utilizó principalmente para el estudio teórico y computacional de los modelos cinemáticos, específicamente en la construcción de tablas de Denavit-Hartenberg (DH), el cálculo de trayectorias y el análisis de resultados.

El reporte está organizado en cinco capítulos. El Marco Teórico cubre los fundamentos sobre cinemática, dinámica, ROS y control de robots. En el capítulo de Desarrollo se describe el diseño del brazo robótico, sus características principales, el proceso seguido para modelar la cinemática directa e inversa, así como su control y simulación. Después, en el capítulo de Resultados, se presentan y analizan el efecto de los programas aplicados al modelo. Finalmente, tenemos las Conclusiones, que resumen los aprendizajes obtenidos.

Este trabajo nos permitió aplicar los conocimientos adquiridos a lo largo del curso en un proyecto práctico y técnico, facilitando la comprensión del comportamiento de los robots en escenarios reales y simulados.