\section{Simulación} \label{sec:simulacion}

\begin{enumerate}
	\item \textbf{Obtención del modelo y exportación desde SolidWorks a URDF.} \\
	El modelo del robot fue seleccionado desde una página web con modelos 3D de brazos robóticos. Se eligió uno sencillo, apropiado para nuestro primer proyecto. En SolidWorks se añadieron los ejes y puntos necesarios para definir correctamente la cinemática. Se utilizó el complemento \textit{sw2urdf} para exportar el modelo al formato URDF, que describe la estructura y articulaciones del robot en XML.
	
	\item \textbf{Uso de Ubuntu y ROS para la simulación.} \\
	Como ROS y Gazebo están optimizados para Linux, se utilizó Ubuntu 20.04 mediante Windows Subsystem for Linux (WSL). Se instaló ROS Noetic, el middleware necesario para controlar y simular el robot. Esto permitió compilar y ejecutar todos los paquetes relacionados con la simulación. Una vez generado el URDF, se accedió desde Visual Studio Code a las carpetas del sistema Ubuntu para editar los archivos y configurar el robot. Ahí se definieron los enlaces, articulaciones, transmisiones y otros parámetros necesarios.
	
	\item \textbf{Importación del proyecto en Ubuntu.} \\
	Se clonó el repositorio del proyecto desde GitHub. Posteriormente, se creó un \textit{workspace} de ROS y se copiaron los archivos del robot en la carpeta \texttt{src/}. Se editaron los archivos \texttt{package.xml} y \texttt{CMakeLists.txt} para añadir autores y dependencias necesarias.
	
	\item \textbf{Modificaciones al archivo URDF.} \\
	Se añadió el eslabón \texttt{world} para fijar el robot al suelo. También se definieron todas las articulaciones y sus límites físicos. Se usó la etiqueta \texttt{<selfCollide>} para evitar colisiones internas, y se añadieron las transmisiones necesarias para simular engranajes o actuadores reales.
	
	\item \textbf{Sincronización de las pinzas (mímica).} \\
	Se utilizó la etiqueta \texttt{<mimic>} para que una pinza copie el movimiento de la otra, simulando un solo actuador que mueve ambas. Esto es común en robots con efectores centrados que utilizan un sistema de engranajes.
	
	\item \textbf{Definir controladores de ROS usando YAML.} \\
	Se creó un archivo \texttt{joint\_trajectory\_controller.yaml} dentro de la carpeta \texttt{config/}, donde se definieron los controladores necesarios para las articulaciones del brazo, las pinzas y el publicador de estados de las juntas. Este archivo es clave para que \texttt{ros\_control} pueda enviar comandos al robot desde MoveIt o Gazebo.
	
	\item \textbf{Crear archivo \textit{launch} del robot en Gazebo.} \\
	Se creó un archivo de lanzamiento dentro de la carpeta \texttt{launch/} del paquete del robot. En este archivo se incluyeron todos los nodos necesarios para lanzar la simulación completa en Gazebo, incluyendo el URDF, los controladores y parámetros correspondientes.
	
	\item \textbf{Ejecutar el archivo \texttt{launch} por primera vez en Gazebo.} \\
	Con el comando \texttt{roslaunch}, se inició la simulación. Esto permitió observar el robot en Gazebo, verificar su estabilidad y comprobar que las articulaciones respondieran correctamente a los comandos.
	
	\item \textbf{Crear paquete con MoveIt Setup Assistant.} \\
	Se utilizó la herramienta gráfica MoveIt Setup Assistant para configurar fácilmente el brazo robótico. A través de esta interfaz, se generó el paquete de MoveIt que incluye el modelo de planificación cinemática, los límites de movimiento y la planificación de trayectorias.
	
	\item \textbf{Probar el paquete creado con MoveIt.} \\
	Finalmente, se lanzó el paquete generado con MoveIt para controlar el robot desde RViz. Esto permitió visualizar el robot en 3D, planear trayectorias y enviar comandos al robot simulado desde una interfaz amigable.
\end{enumerate}