\section{Proceso de Cinemática} \label{sec:proceso_cinematica}

Para el análisis del comportamiento del manipulador, se implementaron los procesos de cinemática utilizando herramientas de simulación en MATLAB, con base en la tabla de parámetros Denavit-Hartenberg (DH) desarrollada a partir del modelo CAD del robot.
Las simulaciones se llevaron a cabo empleando plantillas proporcionadas por el docente, adaptadas a la configuración específica del robot utilizado por el equipo.

A lo largo de esta sección, se llevaron a cabo simulaciones que permitieron obtener, visualizar y validar el comportamiento del efector final y de las articulaciones del robot frente a distintas trayectorias y configuraciones deseadas. Estos procesos resultaron fundamentales para garantizar que el modelo matemático del robot se comportara de forma coherente con su diseño físico.

El análisis se divide en tres etapas principales: cinemática directa, cinemática diferencial y cinemática inversa. Cada una de ellas se desarrolla en detalle en las secciones subsecuentes, junto con los resultados obtenidos y sus respectivas gráficas y animaciones.

% Si no quieres ponerle título al código, puedes dejarlo en blanco.
\begin{matlabcode}{matlab}
	function [q_sol, p_sol] = cinematica_inv(r, p_des, tol, max_iter, alpha, numMuestras)
\end{matlabcode}

\begin{terminal}{bash: Instalar minted en python con pip}
	pip install latexminted==0.5.1
\end{terminal}

\subsection{Cinemática Directa} \label{subsec:cinematica_directa}

La cinemática directa consiste en determinar la posición y orientación del efector final a partir de los valores de las variables articulares del robot. Para el brazo robótico en estudio, esta operación se basa en la tabla de parámetros Denavit-Hartenberg (DH) obtenida del modelo CAD, la cual permite construir las matrices de transformación homogénea necesarias para describir la configuración espacial del efector final. \\

Primero, se actualiza la configuración del robot con los valores articulares actuales:

\begin{matlabcode}{matlab}
	robot = actualizar_robot(robot, q(:,k));
\end{matlabcode}

Luego, se extraen la posición y la matriz de rotación del efector final:

\begin{matlabcode}{matlab}
	posicion(:,k) = robot.T(1:3,4,end);
	R(:,:,k) = robot.T(1:3,1:3,end);
\end{matlabcode}

A continuación, se obtiene la orientación en ángulos de Euler a partir de la matriz de rotación:

\begin{matlabcode}{matlab}
	orientacion(:,k) = rotMat2euler(R(:,:,k), secuencia);
\end{matlabcode}

Con la configuración actualizada, se calcula el Jacobiano geométrico:

\begin{matlabcode}{matlab}
	[Jv(:,:,k), Jw(:,:,k)] = jac_geometrico(robot);
\end{matlabcode}

Finalmente, se calculan la velocidad lineal y angular del efector final:

\begin{matlabcode}{matlab}
	vel_linear(:,k)  = Jv(:,:,k) * (q(:,k)/dt);
	vel_angular(:,k) = Jw(:,:,k) * (q(:,k)/dt);
\end{matlabcode}

Este procedimiento se repite para cada instante de tiempo para obtener la trayectoria completa del efector final.

Los resultados obtenidos con este procedimiento se presentan en la sección de Resultados (\autoref{chap:resultados}).

%Explicar las partes importantes del código de la cinemática directa.
%Para ver los resultados, ir al \autoref{chap:resultados}: Resultados, o determinada figura.
\subsection{Cinemática Diferencial}

La cinemática diferencial describe cómo se relacionan las velocidades articulares del robot con las velocidades lineales y angulares del efector final. Para ello, se utiliza el Jacobiano geométrico, que se compone de dos partes:
\begin{itemize}
	\item $J_v$: Parte traslacional.
	\item $J_w$: Parte rotacional.
\end{itemize}

A continuación se presenta el código que calcula el Jacobiano paso a paso, con su respectiva explicación.

Antes de comenzar, se extraen de la estructura del robot las matrices necesarias para calcular el Jacobiano: la transformación base, las transformaciones globales de cada articulación, el número de grados de libertad, y los tipos de articulaciones.

\begin{matlabcode}{matlab}
	function [Jv, Jw] = jac_geometrico(r)
	A0 = r.A0;
	T = r.T;
	NGDL = r.NGDL;
	tipo = r.tipo;
\end{matlabcode}

Se inicializan las matrices vacías para el Jacobiano traslacional ($J_v$) y rotacional ($J_w$), con ceros.

\begin{matlabcode}{matlab}
	Jv = zeros(3, NGDL);
	Jw = zeros(3, NGDL);
\end{matlabcode}

Se obtiene la posición final del efector desde la última transformación. También se definen el origen y el eje $z$ del sistema base, tomados desde la matriz $A0$.

\begin{matlabcode}{matlab}
	p_end = T(1:3, 4, end);
	p0 = A0(1:3, 4);
	z0 = A0(1:3, 3);
\end{matlabcode}

Se recorre cada articulación del robot. Si es la primera, se usa el sistema base como referencia; en caso contrario, se usa el marco anterior para calcular los vectores necesarios.

\begin{matlabcode}{matlab}
	for i = 1:NGDL
	if i == 1
	p_prev = p0;
	z_prev = z0;
	else
	p_prev = T(1:3, 4, i-1);
	z_prev = T(1:3, 3, i-1);
	end
\end{matlabcode}

Dependiendo del tipo de articulación (revoluta 'r' o prismática 'p'), se calcula la columna $i$ del Jacobiano de forma distinta:
\begin{itemize}
	\item Si es revoluta:
	\begin{itemize}
		\item $Jv(:,i) = z_{i-1} \times (p_{end} - p_{i-1})$.
		\item $Jw(:,i) = z_{i-1}$.
	\end{itemize}
	\item Si es prismática:
	\begin{itemize}
		\item $Jv(:,i) = z_{i-1}$.
		\item $Jw(:,i) = [0; 0; 0]$.
	\end{itemize}
\end{itemize}

\begin{matlabcode}{matlab}
	switch tipo(i)
	case 'r'
	Jv(:, i) = cross(z_prev, (p_end - p_prev));
	Jw(:, i) = z_prev;
	case 'p'
	Jv(:, i) = z_prev;
	Jw(:, i) = [0; 0; 0];
	otherwise
	error('Tipo de junta inválido, debe ser ''r'' o ''p''.');
	end
\end{matlabcode}

\subsection{Cinemática Inversa}

Para determinar las configuraciones articulares que permiten alcanzar posiciones deseadas del efector final, se implementa un método iterativo de cinemática inversa. A continuación se describen las partes principales del código:

\bigskip

Definición de parámetros para el método iterativo:

\begin{matlabcode}{matlab}
	tol = 1e-6;         % Tolerancia para el error en la posición
	max_iter = 100;     % Número máximo de iteraciones permitidas
	alpha = 1.0;        % Factor de corrección en cada iteración
	numSamples = 100;   % Número de configuraciones iniciales aleatorias
\end{matlabcode}

Estos parámetros controlan la precisión, límite de iteraciones y la cantidad de puntos de partida para la búsqueda de soluciones.

\bigskip

Inicialización de matrices para guardar las soluciones obtenidas:

\begin{matlabcode}{matlab}
	q_sol = zeros(robot.NGDL, n);   % Matriz para almacenar las soluciones articulares
	p_sol = zeros(3, n);            % Matriz para almacenar las posiciones calculadas
\end{matlabcode}

Aquí, \texttt{n} corresponde al número de puntos en la trayectoria deseada.

\bigskip

Cálculo iterativo de la cinemática inversa para cada punto de la trayectoria:

\begin{matlabcode}{matlab}
	for i = 1:n
	[q_sol(:,i), p_sol(:,i)] = cinematica_inv(robot, pos_puntos(:,i), tol, max_iter, alpha, numSamples);
	end
\end{matlabcode}

Se recorre cada posición deseada y se calcula la configuración articular correspondiente usando la función \texttt{cinematica\_inv}.

\bigskip

Generación de la trayectoria articular con perfil trapezoidal de velocidad:

\begin{matlabcode}{matlab}
	delta_q = abs(q_sol(:,2) - q_sol(:,1));
	t_final = max(2 * delta_q ./ robot.dqMax);
	numSamples = 201;
	[q, dq, ddq, t, pp] = trapveltraj(q_sol, numSamples, 'Acceleration', robot.ddqMax, 'EndTime', t_final);
\end{matlabcode}

Este bloque calcula un perfil suave para el movimiento articular entre las posiciones calculadas, respetando las velocidades y aceleraciones máximas.

\bigskip

Explicar las partes importantes del código de la cinemática inversa.
Para ver los resultados, ir al \autoref{chap:resultados}: Resultados, o determinada figura.