\section{Ana Claudia Pérez Estupiñán}

Este proyecto me permitió comprender de forma práctica los conceptos fundamentales del análisis cinemático de un manipulador robótico. A través de la implementación de la cinemática directa, diferencial e inversa en MATLAB, pude observar cómo las configuraciones articulares influyen directamente en la posición y orientación del efector final.

Uno de los aprendizajes más importantes fue la interpretación y construcción de la tabla de parámetros Denavit-Hartenberg, ya que representa la base para el modelado matemático del robot. También logré visualizar cómo el Jacobiano geométrico permite relacionar velocidades articulares con velocidades cartesianas, aspecto clave en la cinemática diferencial.

Además, el proceso de cinemática inversa me ayudó a entender la dificultad de encontrar soluciones articulares a partir de una trayectoria deseada, y cómo los métodos iterativos como el descenso del gradiente o el uso de la pseudoinversa del Jacobiano pueden resolverlo.

A lo largo del proyecto, se utilizó un entorno de desarrollo profesional basado en WSL con Ubuntu 20.04, trabajando directamente desde Visual Studio Code. Esto permitió ejecutar herramientas de simulación como ROS, MoveIt para la planificación de trayectorias y Gazebo para la simulación física del robot en un entorno 3D.

La integración de estos elementos me ayudó a conectar los modelos matemáticos desarrollados en MATLAB con simulaciones realistas en Gazebo y visualizaciones en RViz, reforzando la relación entre teoría, programación, y validación práctica. Sin duda, este proyecto fortaleció mi entendimiento del comportamiento de manipuladores robóticos y de las herramientas actuales utilizadas en el desarrollo de robots reales.
