\section{Carlos Francisco Cedano Mendoza}
El proyecto final de la materia de robótica me permitió aplicar y consolidar conocimientos teóricos y prácticos fundamentales en el diseño y simulación de manipuladores robóticos. Partiendo del modelado del robot en SolidWorks y su exportación en formato URDF, logramos integrar el robot en el entorno de simulación Gazebo utilizando ROS sobre Ubuntu, ejecutado a través de WSL para compatibilidad con nuestro sistema operativo Windows.

Durante el desarrollo, realizamos modificaciones esenciales en los archivos URDF para adecuar la estructura del robot y crear el entorno necesario que permitiera su correcto posicionamiento y movimiento dentro de Gazebo. Además, calculamos los torques necesarios para seleccionar motores adecuados en función del peso y materiales del robot, apoyándonos en la tabla de Denavit-Hartenberg para definir correctamente las configuraciones geométricas y cinemáticas del manipulador.

Este proyecto no solo reforzó mi comprensión sobre los conceptos de cinemática y dinámica de robots, sino que también me brindó experiencia práctica en el manejo de herramientas esenciales como SolidWorks, ROS, Gazebo y Ubuntu, así como en la resolución de problemas técnicos propios de la integración de sistemas robóticos.