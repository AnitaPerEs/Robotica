\section{Sebastián Martínez Navarro}
Durante el desarrollo de este proyecto pudimos aplicar de manera práctica los conocimientos adquiridos en el curso de Robótica, particularmente en lo relacionado con la cinemática y dinámica de brazos robóticos.

Uno de los principales problemas fue el modelado cinemático e inverso, ya que se requería una buena interpretación del sistema de coordenadas y del método de Denavit-Hartenberg. También se me dificultó un poco el correr algunos programas en MATLAB ya que en momentos parecía hasta cosa de suerte para que corrieran bien los programas, sin embargo, eso mismo también considero que me ayudó a aprender a solucionar problemas y adentrarme más a la programación de MATLAB.

Otra limitación fue el espacio de mi laptop para descargar los programas así que agradezco que no todos los miembros del equipo tenían que tener todos los software y se logró repartir el trabajo en equipo de manera que todos aportaran en algun área, ya sea modelado, programación o redacción de reportes.

En cuanto a la teoría, fue un reto comprender e implementar las ecuaciones dinámicas del robot, ya que involucran conceptos avanzados como matrices de inercia, fuerzas de Coriolis y fricción, temas en los que no estaba muy familiarizado. 
Otro aspecto importante fue el trabajo en equipo. El uso de herramientas como GitHub y SourceTree, y LaTeX nos permitieron organizarnos mejor y entregar un trabajo más completo, para muchos (yo incluido) fue la primera vez que trabajamos con este tipo de programas, me parece muy bien aprenderlos ya que son herramientas muy comunes en la industria.

