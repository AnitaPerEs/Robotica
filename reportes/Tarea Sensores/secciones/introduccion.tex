\section{Introducción}
En la actualidad, los sensores desempeñan un papel fundamental en diversas áreas de la ingeniería, automatización y tecnología, permitiendo la recolección de datos esenciales para el control y monitoreo de sistemas. Estos dispositivos transforman magnitudes físicas en señales eléctricas que pueden ser interpretadas y procesadas por sistemas electrónicos y de control.

Los sensores se pueden clasificar en diversas categorías según su función y principio de operación. En términos generales, se dividen en sensores internos y externos. Los sensores internos incluyen aquellos que miden variables como posición, velocidad, aceleración y fuerza, empleando tecnologías como encoders, potenciómetros, galgas extensométricas y sensores de efecto Hall. Por otro lado, los sensores externos pueden operar mediante contacto físico o sin contacto, abarcando desde interruptores de límite y transductores de presión hasta sensores de proximidad, ultrasónicos y de visión.

El estudio y comprensión de estos sensores es esencial para su correcta aplicación en sectores como la robótica, la manufactura, la industria automotriz y la automatización de procesos. En esta investigación, se analizarán los distintos tipos de sensores, sus principios de funcionamiento y sus aplicaciones más relevantes en la industria.